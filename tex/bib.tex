\clearpage
\secrel{Ссылки}\secdown

\begin{thebibliography}{99}

\secrel{\F}

    \bibitem{cactus}
    \url{http://www.forth.org.ru/~cactus/library.htm}

    \bibitem{kelly}
    М.Келли, Н.Спайс.\\
    \href{http://www.forth.org.ru/~cactus/files/kelly.rar}{Язык Программирования \F}.\\
    Перевод с английского Е.В. Куркова, Ю.А. Семенова.\\
    Москва, "Радио и связь", 1993

    \bibitem{starting} 
    Лео Броуди.\\
    \textbf{Начальный курс программирования на языке \F}.\\
    - М.: Финансы и статистика, 1990.

    \bibitem{thinking}
    Лео Броуди.\\
    \textbf{Способ мышления\ --- \F.\\Язык и философия для решения задач}.\\
    Перевод с английского С.Н. Дмитренко.\\
    Москва, 1993 г.

    \bibitem{techno}
    \href{http://www.forth.org.ru/~TechnoForth/book.rar}{Учебное пособие по языку \F}.\\
    ИТФ "Технофорт".\\
    Санкт-Петербург, 1993.

    \secdown\secrel{Реализация языка \F}

    \bibitem{baranov}
    С.Н. Баранов, Н.Р. Ноздрунов.\\
    \href{http://www.forth.org.ru/~cactus/files/baranov2.rar}{Язык Форт и его реализации}.\\
    Лениград, "Машиностроение" Ленинградское отделение, 1988.

    \bibitem{bell}
    James R. Bell\\
    \textbf{Threaded Code}
    \href{http://home.claranet.nl/users/mhx/Forth_Bell.pdf}{pdf}\\
    Programming Techniques (ed) R. Morris

    \bibitem{loeliger}
    R. G. Loeliger\\
    \textbf{Threaded Interpretive Languages: Their Design and Implementation}\\
    \href{https://www.amazon.com/Threaded-Interpretive-Languages-Design-Implementation/dp/007038360X}{amazon} |
    \href{http://sinclairql.speccy.org/archivo/docs/books/Threaded_interpretive_languages.pdf}{pdf}

    \secup

\secrel{Мультиплатформенное\\ программирование}

\bibitem{sdl} \verb|libSDL|: низкоуровневое программирование 2D игр

\bibitem{wx} \verb|wxWidgets|: мультиплатформенный графический интерфейс

\secrel{\lin}

\bibitem{x11} Протоколы XOrg X11

\secrel{\win}

\bibitem{frolov3}
библиотека Фролова том 3. WinAPI
\bibitem{frolov4}
библиотека Фролова том 4. Windows GDI Программирование графики

\secrel{микроконтроллеры}

\bibitem{cm3} Процессорное ядро микроконтроллеров Cortex-M3

\secrel{компиляторы}

\bibitem{dragon} Ахо, Сети, Ульман\\
\textbf{Компиляторы: инструменты и технологии} 2 издание

\bibitem{bison}
Степаненко\\
\textbf{Lex/Bison: конструирование компиляторов}

\bibitem{ragel}
Ragel 6.3 руководство пользователя /en/

\bibitem{parsec}
\href{http://theorangeduck.com/page/you-could-have-invented-parser-combinators}{You could have invented Parser Combinators}

\secrel{\py}

\bibitem{lutz}
Марк Лутц\\
\textbf{Изучаем Python.}
\href{https://www.ozon.ru/context/detail/id/156082566/}{Том 1} |
\href{https://www.ozon.ru/context/detail/id/165524776/}{Том 2}

\secrel{\erl/\ex}

\href{https://habr.com/ru/post/450508/}{Джо Армстронг об Elixir, Erlang, ФП и ООП}

\bibitem{sasa}
Саша Юрич\\
\href{https://www.ozon.ru/context/detail/id/164833016/}{Elixir в действии}

\bibitem{beam}
Erik Stenman\\
BEAM:
\href{https://blog.stenmans.org/theBeamBook/}{The Erlang Runtime System}

\secrel{\st}

\bibitem{st}
\st:\\
\url{http://stephane.ducasse.free.fr/FreeBooks/}

\secrel{gamedev}

\bibitem{lamot}
Ламот А., Ратклифф Д., Семинаторе М.\\
\textbf{Секреты программирования игр}\\
М.: Питер, 1995

\bibitem{physx}
Д. Конгер, А. Ламот\\
\textbf{Физика для разработчиков компьютерных игр}\\
2007

\bibitem{zxgame}
А.Капульцевич, И.Капульцевич, А.Евдокимов\\
\textbf{Как написать игру для ZX Spectrum}\\
Питер, 1994

\bibitem{lazyfoo}
\href{https://lazyfoo.net/tutorials/SDL/index.php}{Lazy Foo' \textbf{Beginning Game Programming}}


\bibitem{yamamoto}
Jazon Yamamoto\\
\textbf{The Black Art of Multiplatform Game Programming}

\end{thebibliography}

\secup
