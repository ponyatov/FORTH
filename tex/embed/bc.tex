\clearpage
\secrel{Байт-код}\label{bc}\secdown

\begin{enumerate}
\item метод получения программ минимального объема\note{low-end embedded, IoT}:
\begin{itemize}[nosep]
    \item команды для стековой машины не требуют адресации регистров, достаточно только
    опкода;
    \item команду \verb|call| вложенного вызова\note{самая частая команда в
    \F-коде} можно заменить на короткий 16-битный адрес, с установленным старшим
    битом в первом байте, и сэкономить память на опкоде
    \item опкоды декодируются интерпретатором байт-кода через таблицу указателей
    на \ci-функции, что позволяет подменять и отображать неиспользуемые опкоды
    на процедуры в машинном
    коде\note{\href{http://www.netlib.narod.ru/library/book0001/ch02_01.htm}{\term{шитый
    код} и его разновидности}} (и создавать их используя \term{постфиксный ассемблер})
\end{itemize}
\item независимость от платформы
\item само-модифицирующийся код, без сложных заморочек, не требующий учёта
формата машинного кода, и работы с отдельными битами
\end{enumerate}

Из-за ограничений \term{целевой платформы}, а точнее требований к минимизации
размеров кода, и объема памяти, далее будем рассматривать специализированную
реализацию с урезанным размером \term{машинного слова}, или \term{ячейки} в
\F-терминологии: \textbf{16 бит}.

Такое ограничение выглядит интересно для IoT, или "Интернета вещей". Требуется
не только применение самых дешевых микроконтроллеров, чтобы за те же деньги
получить большее покрытие управлением, и больше данных. В случае обновления
прошивок (программ) следует учитывать, что для беспроводной связи применяются
технологии связи, расчитанные на батарейное питание, короткие пакеты передачи
данных, редкие сеансы связи, и очень низкие скорости (LoRa, NB-IoT,
ZigBee\ldots).

В случае типичного микроконтроллера объем прошивки, которую требуется передать,
иногда с ошибками и повторами, составляет десятки килобайт. При использовании
компиляторов \ci\ необходимо принимать особые меры, чтобы при измененении
исходного кода изменения бинарного файла были минимальны (инкрементная линковка,
компиляция библиотеки \verb|libc| с отключенной LTO-оптимизацией). В некоторых
случаях такая компиляция обновлений прошивки оказывается технически невозможной
(маленький размер Flash-памяти, большие затраты энергии для связи).

Применение байт-кода и языка \F\ может оказаться полезным в IoT:
\begin{itemize}[nosep]
    \item уменьшение объема программ (байт-код vs машинный 32-битный)
    \item удалённое редактирование программ без перепрошивки
    \item вынос простой обработки данных на оконечное устройство
    \item создание расширяемых текстовых протоколов за счет компиляции новых
    команд во время сеанса связи
\end{itemize}

\secup
