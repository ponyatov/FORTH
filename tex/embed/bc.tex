\clearpage
\secrel{Байт-код}\label{bc}\secdown

\begin{enumerate}
\item метод получения программ минимального объема\note{low-end embedded, IoT}:
\begin{itemize}[nosep]
    \item команды для стековой машины не требуют адресации регистров, достаточно только
    опкода;
    \item команду \verb|call| вложенного вызова\note{самая частая команда в
    \F-коде} можно заменить на короткий 16-битный адрес, с установленным старшим
    битом в первом байте, и съэкономить память на опкоде
    \item опкоды декодируются интерпретатором байт-кода через таблицу указателей
    на \ci-функции, что позволяет подменять и отображать неиспользуемые опкоды
    на процедуры в машинном
    коде\note{\href{http://www.netlib.narod.ru/library/book0001/ch02_01.htm}{\term{шитый
    код} и его разновидности}} (и создавать их используя \term{постфиксный ассемблер})
\end{itemize}
\item независимость от платформы
\item само-модифицирующийся код, без сложных заморочек, не требующий учёта
формата машинного кода, и работы с отдельными битами
\end{enumerate}

\secup
