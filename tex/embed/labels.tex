\clearpage
\secrel{Метки вперёд, backpatching}

\noindent
Символьные метки и ссылки вперёд реализуются черед две структуры:
\begin{verbatim}// asm.cpp

// известные метки: 0/1 адрес на метку, переопределения нет
std::map<std::string,CELL>                labels;

// метки вперёд: 1+ адрес на метку
// хранятся адреса параметров команд jmp/call
std::map<std::string,std::vector<CELL>>   forward;
\end{verbatim}

При компиляции вызов/переходов вперед используется \term{backpatching}: так как
адрес перехода ещё не известен, команды компилируются \emph{с фиктивным адресом}
(-1 = 0xFFFF), адреса команд или их параметров накапливаются в структуре вида
\verb|map<string,vector<addr>>|.

При определении каждой новой метки структура \verb|forward{}| проверяется на
наличие списа сохранённых адресов, по которым выполняется корректировка ранее
скомпилированного кода: параметры команд перехода переписываются на новый
известный адрес (\term{ссылка вперёд}), затем запись о найденной метке удаляется
из \verb|forward{}|.

\smallskip
Одновременно с этим адреса всех меток в процессе \term{прохода компиляции}
добавляются в таблицу известных меток \verb|labels{}|. При компиляции команд
перехода на символьную метку выполняется поиск \emph{единственного} адреса по
этой таблице, и команда компилируются сразу с известным адресом (\term{ссылкой
назад}).

\fig{tmp/backpatch.pdf}{width=\textwidth}

\begin{verbatim}
#include <map>
#include <vector>
std::map<std::string,CELL>                labels;
std::map<std::string,std::vector<CELL>>   forward;
\end{verbatim}
\clearpage
\begin{verbatim}
void CL(std::string* l, CELL a)               // Compile Label
{
  if (labels.find(*l) == labels.end()) {      // if unknown
      if (forward.find(*l) == forward.end())  // forwards:
          forward[*l] = std::vector<CELL>();  // new addr[]
      forward[*l].push_back(a);               // append
      CC(-1);                                 // addr:-1
  } else 
      CC(labels[*l]);                         // addr:known
}
\end{verbatim}
Код \verb|data.find(some) == data.end()|\ --- проверка на отсутствие элемента в
контейнере данных, один из кодовых шаблонов \cpp.
\clearpage
\begin{verbatim}
void LL(std::string* l, CELL a)         // Add Label
{
  if (labels.find(*l) == labels.end()) 
    labels[*l] = a;                     // append
    auto vec = forward.find(*l);        // scan fwd[l]->addr[]
    if ( vec != forward.end()) {        // resolve forwards
        for (auto addr:vec->secon
            *((UCELL*)&M[addr]) = a;    // backpatch
        forward.erase(*l);              // drop from fwd
    }
  } else
      yyerror(l->c_str());      // existing label redefinition
}
\end{verbatim}
