\clearpage
\secrel{flex лексер}

Для синтаксического разбора в байт-ассемблере применяется классическая связка
генераторов:
\begin{description}
    \item[lex/flex] генерирует код \term{лексера}: на основе набора
    \term{регулярных выражений} создается файл \verb|tmp/lexer.c| c
    \term{конечным автоматом}, который выбирает из исходного кода символы
    поштучно, и группирует их в \term{лексемы}\note{структура включает строку,
    тип лексемы, номер строки в исходном файле}
    \item[yacc/bison] генерирует код \term{парсера}: синтаксис может быть
    достаточно сложным, как минимум инфиксным, со скобками, определением
    структур данных, с проверкой допустимых форматов команд ассемблера, и т.д.
    Сгенерированный парсер использует внешний лексер через вызов функции
    \verb|yylex()|, и переменные \verb|yytext|, \verb|yylineno|.
\end{description}

\begin{verbatim}// asm.hpp
extern int  yylex();        // интерфейс лексера
extern char *yytext;        // строка лексемы
extern int  yylineno;       // текущая строка исходного кода
\end{verbatim}

Оба инструмента позволяют вставлять произвольный код на \ci/\cpp, выполняемый
при срабатывании каждого правила. Также поддерживаются переключения режима,
например общий синтаксис, и разбор строк работают в разных режимах.

\begin{verbatim}// asm.hpp
extern int yyparse();                   // интерфейс парсера
extern void yyerror(const char *msg);   // error callback
#include "tmp/parser.hpp"
\end{verbatim}

\verb|yacc| предоставляет для лексера описания типов лексем, в виде их
целочисленных кодов, и описания структур для их хранения.

\begin{verbatim}// asm.lex
%{
    #include "asm.hpp"
%}

%option noyywrap yylineno

%%
\end{verbatim}
