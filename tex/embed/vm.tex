\clearpage
\secrel{Виртуальная машина:\\интерпретатор байт-кода}\label{vm}\secdown

\begin{verbatim}
src/
    vm.c vm.h config.h
    all.c all.h
    linux.c linux.h win32.c win32.h
    cortex.c cortex.h
bin/
    vm[.exe]
\end{verbatim}

\noindent
VM написана на ANSI \ci\ максимально переносимо, чтобы можно было её
скомпилировать под любую существующую платформу, по крайней мере
системо-независимый набор команд и каналов ввода/вывода. Для каждой целевой
платформы также включен файл системо-зависимого кода.

\clearpage
\begin{verbatim}// config.h
                            // ячейка (машинное слово)
#define CELL int16_t
                            // объем основной памяти, байт
#define Msz 0x1000
                            // размер стека возвратов, ячеек
#define Rsz 0x100
                            // размер стека данных, ячеек
#define Dsz 0x10
\end{verbatim}

\noindent
Образ памяти и указатель компиляции используются одновременно и в ВМ, и в
байт-ассемблере:

\begin{verbatim}// all.c
BYTE  M[Msz];               // основная память форт-машины
UCELL Cp=0;                 // указатель компиляции
\end{verbatim}

\clearpage
\begin{verbatim}// vm.c
UCELL Ip=0;                     // указатель инструкций
UCELL R[Rsz]; UCELL Rp=0;       // стек возвратов
 CELL D[Dsz]; BYTE  Dp=0;       // стек данных < 255 ячеек
\end{verbatim}

\begin{verbatim}// vm.c
int main(int argc, char *argv[]) {
    // проверка аргументов командной строки: указать .bc файл
    assert(argc>=2); args(argc,argv);
    // загрузка байт-кода
    FILE *img = fopen(argv[1],"rb"); assert(img);
    assert(fread(M,Msz,1,img)); fclose(img);
    // запуск интерпретатора
    interpreter();
\end{verbatim}

\secup
