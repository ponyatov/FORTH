\clearpage
\secrel{Виртуальная машина:\\интерпретатор байт-кода}\label{vm}\secdown

\begin{verbatim}
src/
    vm.c vm.h config.h
    linux.c linux.h win32.c win32.h
    cortex.c cortex.h
bin/
    vm[.exe]
\end{verbatim}

\noindent
VM написана на ANSI \ci\ максимально переносимо, чтобы можно было её
скомпилировать под любую существующую платформу, по крайней мере
системо-независимый набор команд и каналов ввода/вывода. Для каждой целевой
платформы также включен файл системо-зависимого кода.

\clearpage
\begin{verbatim}// config.h
                            // ячейка (машинное слово)
#define CELL int16_t
                            // объем основной памяти, байт
#define Msz 0x1000
                            // размер стека возвратов, ячеек
#define Rsz 0x100
                            // размер стека данных, ячеек
#define Dsz 0x10
\end{verbatim}

\clearpage
\secrel{Память \F-ВМ}
\begin{verbatim}// vm.c
BYTE  M[Msz];                   // основная память форт-машины
UCELL Cp=0;                     // указатель компиляции
#ifdef VM
UCELL Ip=0;                     // указатель инструкций
UCELL R[Rsz]; UCELL Rp=0;       // стек возвратов
 CELL D[Dsz]; BYTE  Dp=0;       // стек данных < 255 ячеек
#endif // VM
\end{verbatim}

Некоторый код, и структуры данных типа образа памяти и указателя компиляции,
используются одновременно и в виртуальной машине, и в байт-ассемблере. Другие
фрагменты кода наоборот должны отличаться, например функция
\verb|main()|\ \ref{init}.

\clearpage
Поэтому при компиляции, в исходном коде используются два макро-определения, и
условная компиляция через \verb|#ifdef|:
\begin{itemize}
    \item \verb|#ifdef ASM|
    \item \verb|#ifdef VM|
\end{itemize}

\begin{verbatim}// Makefile

bin/vm: $(C) $(H)
  $(TCC) $(CFLAGS) -D$(OS) -DVM  -o $@ src/vm.c src/$(OS).c

bin/asm: $(C) $(H)
  $(CXX) $(CFLAGS) -D$(OS) -DASM -o $@ src/vm.c src/$(OS).c \
                     src/asm.cpp tmp/lexer.cpp tmp/parser.cpp
\end{verbatim}

\clearpage
\secrel{Запуск системы}\label{init}
\begin{verbatim}// vm.c

#ifdef VM
int main(int argc, char *argv[]) {
    // проверка аргументов командной строки: указать .bc файл
    assert(argc>=2); args(argc,argv);
    // загрузка байт-кода
    FILE *img = fopen(argv[1],"rb"); assert(img);
    assert(fread(M,Msz,1,img)); fclose(img);
    // запуск интерпретатора
    interpreter();
}
#endif // VM
\end{verbatim}

\secup
