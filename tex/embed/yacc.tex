\clearpage
\secrel{bison парсер/компилятор}

\begin{verbatim}// asm.yacc
%{
    #include "asm.hpp"
%}
\end{verbatim}
Директива включает генерацию \verb|tmp/parser.hpp| с описаниями токенов для лексера:
\begin{verbatim}
%defines %union { BYTE b; CELL c; std::string* s; }
\end{verbatim}
% \clearpage
\noindent
Полный перечень токенов с указанием их типов:
\begin{verbatim}
%token <b> op0      %token <b> op1      %token <c> num
%token colon        %token <s> sym
\end{verbatim}
Рекурсивное стартовое правило: пустой файл, или последовательность команд:
\begin{verbatim}
%%
REPL :
REPL : REPL cmd
\end{verbatim}
Доступны два вида команд: одиночная, и команда с параметром:
\begin{verbatim}
cmd : op0       { CB($1);            } // byte:op
cmd : op1 num   { CB($1); CC($2);    } // byte:op cell:number
cmd : op1 sym   { CB($1); CL($2,Cp); } // byte:op cell:addr|-1
cmd : sym colon { LL($1,Cp);         } // Cp -> labels/forward
\end{verbatim}
Компиляция выполняется набором функций, которые принимают значение байта или
ячейки, и добавляют их в буфер компиляции \verb|M[Cp<Msz]|.

\clearpage
\begin{verbatim}// asm.cpp
void CB(BYTE b) {                   // Compile Byte:
    M[Cp++] = b; assert(Cp<Msz);    // скомпилировать байт
}
void CC(CELL c) {                   // Compile Cell
    *((CELL*)&M[Cp]) = c;           // только для
    Cp += 2; assert(Cp<Msz);        // little-endian
}
\end{verbatim}
Для компиляции ячейки используеся хитрое заклинание с указателями, которое
правильно работает только для little-endian архитектур. Удачно, что для
большинства популярных архитектур порядок байт совпадает, поэтому запуск на x86
скомпилирует \textbf{сначала младший байт} ячейки.
