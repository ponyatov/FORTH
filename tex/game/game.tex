\clearpage
\secrel{byte-coded gamedev}\secdown

\clearpage
\term{gamedev}\note{разработка компьютерных игр}\ очень обширная область
программирования, хорошо подходит для начинающих,
так как сочетает требования по многим темам в программировании, с нескушными
обоями, и грабежом корованов. Стимулирует изучение программирования со
структурами данных и алгоритмами, прикладные технологии\note{не забывайте писать
платформы для сетевых игр на 100500 игроков, и экономические статегии}, и лечит
выгорание.

Также область разработки простых indy\note{игра, написанная одним--двумя
разработчиками, без студий и толп дизайнеров}\ игр хорошо подходит для
иллюстрации различных технологий програмимрования. Соответственно, здесь мы
фокусируемся на том, чтобы показать применение байт-ассем\-блера,
микроинтерпретаторов и \F-подходов.

Вообще применение \term{скриптовых движков}, и интерпретаторов\ --- всегда
использовалось игроделами, ещё с самых ранних времён текстовых игр на зеленых
терминалах. Решалось как обычно две основных проблемы:
\begin{enumerate}[nosep]
    \item \term{очень маленькие объемы ОЗУ}\note{на некоторых старых компьютерах
    80х единицы, максимум пара десятков Килобайт}, см. \ref{embed}:
    различные разновидности байт- и шитого кода позволяют упаковать
    последовательности достаточно сложных операций, традиционно применяемых при
    написании игр, в очень компактную форму.
    \item \term{создание SDL-языка максимально высокого уровня}, специально
    заточенного на описание логики и механики игр\note{одной, класса игр, или
    язык более или менее универсального \term{игрового движка}}
    \item \term{обеспечение переносимости}\ игры на как можно большее число
    платформ, типов компьютеров, и операционных систем
\end{enumerate}

\medskip\noindent
Последние два пункта делают применение скриптового движка практически
обязательным для всех игроделов от Васи-школотрона до высшей мировой лиги:
меньше усилий, больше денег.

\clearpage
\secrel{libSDL}\secdown

\href{https://lazyfoo.net/tutorials/SDL/index.php}{Lazy Foo' \textbf{Beginning Game Programming}}

\bigskip\noindent Библиотека \textbf{Simple DirectMedia Layer} предоставляет
достаточно низкоуровневый, но одновременно переносимый интерфейс к базовому
набору периферии и функций, необходимых практически любой компьютерной игре в
старом стиле компьютеров 80х, на который мы ориентируемся в этой главе.

Мы не будем гнаться за 3D, и тем более использовать GPU\ --- в игре главное идея
и геймплей, а не навороты очередной видимокарты. Ну и ретро-гейминг
\cite{zxgame} тоже интересное направление, которое еще дает и дополнительную
переносимость на устройства без OpenGLES.

\clearpage
\secrel{Hello}

\url{https://lazyfoo.net/tutorials/SDL/01_hello_SDL/index.php}

\secup


\secup
