% Universal LaTeX headers for e-book publications
\documentclass[oneside,10pt]{book}
%% mobile phone optimized: Honor 3C 108 mm x 62 mm scaled 1.1 = 118.8 x 68.2
\usepackage[paperwidth=118.8mm,paperheight=68.2mm,margin=2mm]{geometry}
%% font setup for screen reading
\renewcommand{\familydefault}{\sfdefault}\normalfont
%% hyperlinks pdf style
\usepackage[unicode,colorlinks=true]{hyperref}
%% fix heading styles for tiny paper
% \usepackage{titlesec} % breaks .pdf
% \titleformat{\chapter}{\Large\bfseries}{\thechapter.}{1em}{}
% \titleformat{\section}{\large\bfseries}{\thesection.}{1em}{}
%% fix first blank page
\usepackage{atbegshi}% http://ctan.org/pkg/atbegshi
\AtBeginDocument{\AtBeginShipoutNext{\AtBeginShipoutDiscard}}

% Cyrillization
%% \usepackage[T1,T2A]{fontenc}
\usepackage[utf8]{inputenc}
\usepackage[english,russian]{babel}
\usepackage{indentfirst}

% graphics
\usepackage[pdftex]{graphicx}
\newcommand{\fig}[2]{{\centering\noindent\includegraphics[#2]{#1}}}

% xcolor fixes
\usepackage{xcolor}
\definecolor{red}{rgb}{0.5,0,0}		    % R
\definecolor{green}{rgb}{0,0.5,0}	    % G
\definecolor{blue}{rgb}{0,0,0.5}	    % B
\definecolor{magenta}{rgb}{0.5,0,0.5}	% magenta

% relative sectioning
\usepackage{ifthen}
\newcounter{secdepth}\setcounter{secdepth}{0}
\newcommand{\secup}{\addtocounter{secdepth}{1}}
\newcommand{\secdown}{\addtocounter{secdepth}{-1}}
\newcommand{\secrel}[1]{
\ifthenelse{\equal{\value{secdepth}}{0}}{\part{#1}}{}
\ifthenelse{\equal{\value{secdepth}}{-1}}{\chapter{#1}}{}
\ifthenelse{\equal{\value{secdepth}}{-2}}{\section{#1}}{}
\ifthenelse{\equal{\value{secdepth}}{-3}}{\subsection{#1}}{}
\ifthenelse{\equal{\value{secdepth}}{-4}}{\subsubsection{#1}}{}
}
\newcommand{\secly}[1]{
\section*{#1}
\addcontentsline{toc}{section}{#1}
}
\newcommand{\subsecly}[1]{
\subsection*{#1}
\addcontentsline{toc}{subsection}{#1}
}

% misc
\newcommand{\email}[1]{$<$\href{mailto:#1}{#1}$>$}
\renewcommand{\emph}[1]{\textit{#1}}
\newcommand{\term}[1]{\textcolor{green}{#1}}
\newcommand{\note}[1]{\,\footnote{\ #1}}
%% [nosep] option in lists/enums
\usepackage{enumitem}

%% languages
\newcommand{\ci}{\texttt{Си}}
\newcommand{\py}{\texttt{Python}}
\newcommand{\F}{\texttt{Форт}}

%% OSs
\newcommand{\lin}{\texttt{Linux}}
\newcommand{\win}{\texttt{Windows}}
